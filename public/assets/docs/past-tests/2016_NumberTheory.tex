\documentclass[10pt]{article}
\usepackage{amsmath, amssymb, amsthm}
\usepackage[top=2cm, left = 2cm, right = 2cm, bottom = 3cm]{geometry}
\usepackage[pdftex]{graphicx}
\usepackage{asymptote}
\usepackage{fancyhdr}
\pagestyle{fancy}
\rhead{}
\chead{\includegraphics[scale=0.12]{CMIMC-header.png}}
\lhead{}
\setlength{\headheight}{43pt}
\rfoot{}
\cfoot{}
\lfoot{}
\addtolength\footskip{-1cm}
\begin{document}\thispagestyle{empty}
\begin{center}

\vspace*{90pt}

\includegraphics[scale=0.23]{CMIMC-header.png}

\includegraphics[scale=0.35]{nt-header.png}

\vspace{1.6in}

\includegraphics[scale=0.20]{instruction-header.png}
\noindent\rule{17.7cm}{2pt}
\end{center}

\vspace{10pt}

\begin{enumerate}
\large
\item Do not look at the test before the proctor starts the round.

\item This test consists of 10 short-answer problems to be solved in 60 minutes.
	Each question is worth one point.

\item Write your name, team name, and team ID on your answer sheet. Circle the
	subject of the test you are currently taking.

\item Write your answers in the corresponding boxes on the answer sheets.

\item No computational aids other than pencil/pen are permitted.

\item All answers are integers.

\item If you believe that the test contains an error, submit your protest in writing to Porter 100.
\end{enumerate}

\newpage

\begin{center}
\huge\textbf{Number Theory}\normalsize

\vspace{3pt}
\end{center}

\begin{enumerate}
\setlength{\itemsep}{3pt}

\item David, when submitting a problem for CMIMC, wrote his answer 
	as $100\tfrac xy$, where $x$ and $y$ are two positive integers
	with $x<y$. Andrew interpreted the expression as a product of two
	rational numbers, while Patrick interpreted the answer as a mixed
	fraction. In this case, Patrick's number was exactly double Andrew's!  
	What is the smallest possible value of $x+y$? %David Altizio

\item Let $a_1$, $a_2$, $\ldots$ be an infinite sequence of integers such that
	$k$ divides $\gcd(a_{k-1},a_k)$ for all $k\geq 2$.  Compute the smallest possible value of $a_1+a_2+\cdots+a_{10}$. %David Altizio

\item How many pairs of integers $(a,b)$ are there such that $0\leq a < b \leq 100$ and such that $\tfrac{2^b-2^a}{2016}$ is an integer? %Cody Johnson

\item For some positive integer $n$, consider the usual prime factorization \[n =
	\displaystyle \prod_{i=1}^{k} p_{i}^{e_{i}}=p_1^{e_1}p_2^{e_2}\ldots p_k^{e_k},\] where $k$ is the number of
	primes factors of $n$ and $p_{i}$ are the prime factors of $n$. Define
	$Q(n), R(n)$ by
	\[
		Q(n) = \prod_{i=1}^{k} p_{i}^{p_{i}} \text{ and } R(n) =
		\prod_{i=1}^{k} e_{i}^{e_{i}}.
	\]
	For how many $1 \leq n \leq 70$ does $R(n)$ divide $Q(n)$?

\item Determine the sum of the positive integers $n$ such that there exist primes
	$p,q,r$ satisfying $p^{n} + q^{2} = r^{2}$. %Andrew Kwon

\item Define a \textit{tasty residue} of $n$ to be an integer $1<a<n$ such that there exists an integer $m>1$ satisfying \[a^m\equiv a\pmod n.\] Find the number of tasty residues of $2016$. %Andrew Kwon

\item Determine the smallest positive prime $p$ which satisfies the congruence \[p+p^{-1}\equiv 25\pmod{143}.\] Here, $p^{-1}$ as usual denotes multiplicative inverse. %David Altizio

\item Given that
	\[
		\sum_{x=1}^{70}  \sum_{y=1}^{70} \frac{x^{y}}{y} =
		\frac{m}{67!}
	\]
	for some positive integer $m$, find $m \pmod{71}$.


\item Compute the number of positive integers $n \leq 50$ such that there exist
	distinct positive integers $a,b$ satisfying 
	\[
		\frac{a}{b} +\frac{b}{a} = n \left(\frac{1}{a} + \frac{1}{b}\right).
	\]
	%David Altizio

\item Let $f:\mathbb{N}\mapsto\mathbb{R}$ be the function
	\[f(n)=\sum_{k=1}^\infty\dfrac{1}{\operatorname{lcm}(k,n)^2}.\] It is
	well-known that $f(1)=\tfrac{\pi^2}6$.  What is the smallest positive
	integer $m$ such that $m\cdot f(10)$ is the square of a rational multiple of $\pi$? %Cody Johnson

\end{enumerate}
\end{document}
