\documentclass[10pt]{article}
\usepackage{amsmath, amssymb, amsthm}
\newtheorem{lemma}{Lemma}
\usepackage[top=2cm, left = 2cm, right = 2cm, bottom = 3cm]{geometry}
\usepackage[pdftex]{graphicx}
\usepackage{asymptote}
\usepackage{fancyhdr}
\newcommand{\N}{\mathbb{N}}
\pagestyle{fancy}
\rhead{}
\chead{\includegraphics[scale=0.1]{../CMIMC-header-2018.png}}
\lhead{}
\setlength{\headheight}{43pt}
\rfoot{}
\cfoot{}
\lfoot{}
\newcommand{\proposed}[1]
{
\vspace{5pt}
\noindent\textit{Proposed by #1}
}
\newcommand{\solution}
{
\vspace{5pt}
\noindent\textit{Solution.}\qquad
}
\begin{document}

\begin{center}
\huge\textbf{Algebra Solutions Packet}\normalsize

\vspace{3pt}
\end{center}

\begin{enumerate}

\item Misha has accepted a job in the mines and will produce one ore each day. At the market, he is able to buy or sell one ore for \$3, buy or sell bundles of three wheat for \$12 each, or \textit{sell} one wheat for one ore. His ultimate goal is to build a city, which requires three ore and two wheat. How many dollars must Misha begin with in order to build a city after three days of working?

\proposed{Patrick Lin}

\solution Suppose Misha begins with $k$ dollars; after three days of working he will have $k+9$ dollars (without loss of generality he sells his ore). The cheapest way to get three ore and two wheat is to purchase three wheat, trade one for an ore, then purchase two more ore; this costs $12 + 3 + 3 = 18$ dollars, and so the minimal $k$ is $\boxed{9}$.

%\item For positive real numbers $r$ and $s$, define $r \star s = \sqrt{rs}$. If
%\[x=20\star(18\star(20\star(18\star \dots))),\]
%what is the smallest integer power of $x$ greater than 1?

\item Suppose $x>1$ is a real number such that $x+\tfrac 1x = \sqrt{22}$.  What is $x^2-\tfrac1{x^2}$?

\proposed{David Altizio}

\solution Note that \[\left(x + \frac1x\right)^2 - \left(x - \frac1x\right)^2 = x^2 + 2 + \frac{1}{x^2} - \left(x^2 - 2 + \frac1{x^2}\right) = 4.\] Thus $x - \tfrac1x = \sqrt{(\sqrt{22})^2 - 4} = 3\sqrt 2$.  It follows that \[x^2 - \frac{1}{x^2} = \left(x - \frac1x\right)\left(x + \frac 1x\right) = \boxed{6\sqrt{11}}.\] Alternatively, it is not terribly difficult to compute $x=\frac12(\sqrt{22}+\sqrt{18})$ and then perform the necessary arithmetic to arrive at the value of $x^2-\tfrac{1}{x^2}$.

\item Let $P(x)=x^2+4x+1$.  What is the product of all real solutions to the equation $P(P(x))=0$?

\proposed{David Altizio}

\solution{Let $t$ be a real solution to $P(P(x)) = 0$.  Note that $P(P(t)) = 0$ implies that $P(t)$ is a root of the polynomial $P$.  These roots can be computed to be $-2\pm\sqrt 3$ by any method of your choice, so $t$ must satisfy either $t^2+4t+1 = -2+\sqrt 3$ or $t^2 + 4t + 1 = -2-\sqrt 3$; these are equivalent to $t^2 + 4t + 3-\sqrt 3 = 0$ and $t^2 + 4t + 3+\sqrt 3 = 0$ respectively.  Now it is easy to see that the minus solution has no real roots (since $3+\sqrt 3 > 4$), while the plus solution has two real roots with product $\boxed{3-\sqrt 3}$.}

%\item Among all ordered pairs of real numbers $(a,b)$ satisfying $a^2+ab+b^2=6$, compute the largest possible value of $a$.

\item 2018 little ducklings numbered 1 through 2018 are standing in a line, with each holding a slip of paper with a nonnegative number on it; it is given that ducklings 1 and 2018 have the number zero.  At some point, ducklings 2 through 2017 change their number to equal the average of the numbers of the ducklings to their left and right.  Suppose the new numbers on the ducklings sum to 1000.  What is the maximum possible sum of the original numbers on all 2018 slips?

\proposed{David Altizio}

\solution Let $n=2018$ for simplicity.  Denote the numbers on the slips of paper as $A_1,A_2,\ldots, A_{n}$, with $A_1 = A_{n} = 0$, and set $S=A_1 + \cdots + A_{2018}$.  Now remark that after taking averages, the ducklings now have the numbers \[0,\frac{A_3}2,\frac{A_2 + A_4}2,\cdots,\frac{A_{n-1} + A_{n-3}}2,\frac{A_{n-2}}2,0.\] Each variable except for $A_2$ and $A_{n-1}$ appears twice, and so the sum of the numbers now is \[1000 = \frac{A_2}2 + A_3 + \cdots + A_{n-2} + \frac{A_{n-1}}2 = S - \frac{A_2 + A_{n-1}}2.\] Now since each of the $A_i$ is nonnegative we have $A_2 + A_{n-1}\leq S$, and so $1000\geq S - \frac{S}2 = \frac{S}2$, which yields $S\leq\boxed{2000}$.  Equality holds when all of $A_3$ through $A_{2016}$ are zero.

\item Suppose $a$, $b$, and $c$ are nonzero real numbers such that \[bc+\frac1a = ca+\frac2b = ab+\frac7c = \frac1{a+b+c}.\] Find $a+b+c$.

\proposed{David Altizio}

\solution Note that the system of equations rearranges to \[abc+1=\frac{a}{a+b+c},\quad abc+2=\frac{b}{a+b+c},\quad abc+7 = \frac{c}{a+b+c}.\] Now adding these together yields \[3abc + 10 = \frac{a+b+c}{a+b+c} = 1\quad\Rightarrow\quad abc = -3.\] Thus $\frac ab = \frac{abc+1}{abc+2} = 2$ and $\frac bc = \frac{abc+2}{abc+7} = -\frac14$.  This in turn implies $a=2b$ and $c=-4b$.  Then $bc+\frac1a = \frac1{a+b+c}$ becomes \[-4b^2 + \frac1{2b} = \frac{1}{b + 2b - 4b} = -\frac1b\quad\Rightarrow\quad b = \frac{\sqrt[3]3}2,\] and so $a+b+c = \boxed{-\tfrac{\sqrt[3]3}2}$.

\item We call $\overline{a_n\ldots a_2}$ the Fibonacci representation of a positive integer $k$ if \[k = \sum_{i=2}^n a_i F_i,\] where $a_i\in\{0,1\}$ for all $i$, $a_n=1$, and $F_i$ denotes the $i^{\text{th}}$ Fibonacci number ($F_0=0$, $F_1=1$, and $F_i=F_{i-1}+F_{i-2}$ for all $i\ge2$). This representation is said to be \textit{minimal} if it has fewer 1’s than any other Fibonacci representation of $k$. Find the smallest positive integer that has eight ones in its minimal Fibonacci representation. 

\proposed{Darshan Chakrabarti}

\solution A Fibonacci representation is minimal if and only if it has no two consecutive ones: If there were an $i\ge1$ such that $a_{i+1}=a_i=1$, choose the largest such $i$, so that $a_{i+2}=0$. Replace $a_{i+2}$ with $1$ and $a_{i+1},a_i$ with $0$ to get a representation with strictly fewer ones, contradicting minimality. The other direction follows by uniqueness of Zeckendorf decompositions.

Now let $\overline{a_n\ldots a_1}$ be the minimal Fibonacci representation of the answer, so that $a_n=1$ and $a_i:=0$ for all $i>n$.

There is no $n-1>i\ge2$ such that $a_{i+1}=a_i=0$: If there were, then choose the largest such $i$, so that $a_{i+2}=1$. Replace $a_{i+2}$ with $0$ and $a_{i+1}$ with $1$ to get a minimal Fibonacci representation of a strictly smaller positive integer.

Thus, the answer is either $\ldots10101$ or $\ldots01010$. The latter is clearly larger, so the answer is \[F_2+F_4+\dots+F_{16}=\sum_{k=1}^8F_{2k}=\sum_{k=1}^8(F_{2k+1}-F_{2k-1})=F_{17}-F_1=1597-1=\boxed{1596}.\] 

\item Compute
\[\sum_{k=0}^{2017}\dfrac{5+\cos\left(\frac{\pi k}{1009}\right)}{26+10\cos\left(\frac{\pi k}{1009}\right)}.\]

\proposed{Misha Ivkov}

\solution Let $r_k = 5 + e^{\pi ik/1009}$.  Then \begin{align*}\frac{5 + \cos(\frac{\pi k}{1009})}{26 + 10\cos(\frac{\pi k}{1009})} &= \frac{5 + \cos(\frac{\pi k}{1009})}{25 + 10\cos(\frac{\pi k}{1009}) + \cos^2(\frac{\pi k}{1009}) + \sin^2(\frac{\pi k}{1009})}\\&=\frac{5 + \cos(\frac{\pi k}{1009})}{(5 + \cos(\frac{\pi k}{1009}))^2 + \sin^2(\frac{\pi k}{1009})}=\Re\left(\frac{r_k}{|r_k|^2}\right) = \Re\left(\frac{1}{\bar{r_k}}\right).\end{align*} Thus the desired sum is just the real part of the complex number sum \[\sum_{k=0}^{2017}\frac{1}{\bar{r_k}} = \sum_{k = 0}^{2017}\frac{1}{r_{2018-k}} = \sum_{k=0}^{2017}\frac{1}{r_k}.\] To compute this, remark that the $r_k$ are the roots of the polynomial $(z-5)^{2018} = 1$.  Hence the desired sum is the sum of the reciprocals of this polynomial, which by Vieta's is \[-\frac{[z^{2017}]((z-5)^{2018} - 1)}{[z^{2018}]((z-5)^{2018} - 1)} = \boxed{\frac{2018\cdot 5^{2017}}{5^{2018} - 1}}.\]

\begin{center}OR\end{center}

\solution Let $P(x)=x^{2018}-1$, and let $z_k=e^{\frac{2i\pi k}{2018}}$ correspond to the roots of this polynomial. Then the sum is \begin{align*}\sum_{k=0}^{2017}\dfrac{5+\cos\left(\frac{\pi k}{1009}\right)}{26+10\cos\left(\frac{\pi k}{1009}\right)}&=\sum_{k=0}^{2017}\frac{5+\frac{z_k+\frac{1}{z_k}}{2}}{26+10\frac{z_k+\frac{1}{z_k}}{2}}=\sum_{k=0}^{2017}\frac{z_k^2+10z_k+1}{2(z_k+5)(5z_k+1)}\\&=\sum_{k=0}^{2017}\frac{1}{10}+\frac{1}{2}\sum_{k=0}^{2017}\frac{1}{z_k+5}-\frac{1}{50}\sum_{k=0}^{2017}\frac{1}{z_k+\frac{1}{5}}=\frac{2018}{10}-\frac{P'(-5)}{2P(-5)}+\frac{P'\left(-\frac{1}{5}\right)}{50P\left(-\frac{1}{5}\right)}\end{align*} which, with some computation, yields the desired answer of $\boxed{\tfrac{2018\cdot 5^{2017}}{5^{2018}-1}}$.

\item Suppose $P$ is a cubic polynomial satisfying $P(0) = 3$ and \[(x^3 - 2x + 1 - P(x))(2x^3 - 5x^2 + 4 - P(x))\leq 0\] for all $x\in\mathbb R$.  Determine all possible values of $P(-1)$.

\proposed{David Altizio}

\solution Let $Q(x) = x^3 - 2x + 1$ and $R(x) = 2x^3 - 5x^2 + 4$, so that the inequality in question becomes $(Q(x) - P(x))(R(x) - P(x))\leq 0$.  The crucial claim is that there exists $t\in(0,1)$ such that \[P(x) \equiv tQ(x) + (1-t)R(x).\] To prove this, we first remark that the graphs of $Q(x)$ and $R(x)$ intersect at three points.  Indeed, $R(x) = Q(x)$ is equivalent to $x^3 - 5x^2 + 2x + 4 = 0$; denote this polynomial by $S(x)$.  Now $S(-1) = -3$, $S(0) = 4$, $S(2) = -4$, and $S(10) = 524$, and so indeed by IVT we conclude that $S$ has three real roots. Now let $r_1$, $r_2$, and $r_3$ be the roots of $S$, so that $Q(r_i) = R(r_i) = r_0$ for some $r_0\in\mathbb R$.  Then plugging $x=r_i$ into the given inequality yields \[(r_0 - P(r_i))(r_0 - P(r_i)) = (r_0 - P(r_i))^2 \leq 0,\] and so by the Trivial Inequality we have $P(r_i) = Q(r_i) = R(r_i)$.  As a result, for some $c$ we have \[P(x) - Q(x) = c(x-r_1)(x-r_2)(x-r_3) = c(R(x) - Q(x)),\] and so there exist constants $\alpha$ and $\beta$ such that \[P(x) = \alpha Q(x) + \beta R(x)\] with $\alpha + \beta = 1$.  It remains to check that $\alpha$ and $\beta$ are positive, but this is easy: after all, the leading coefficient of $P$ must be strictly between the leading coefficients of $Q$ and $R$, or else the inequality breaks as $x\to\pm\infty$.

\par The rest of the problem is easy.  Plugging $x=0$ yields \[3 = t\cdot 1 + (1-t)\cdot 4 = 4 - 3t;\] this yields $t = \frac13$.  Thus \[P(x) = \frac13(x^3 - 2x + 1) + \frac23(2x^3 - 5x^2 + 4)\] and so $P(-1) = \tfrac13\cdot 2 + \tfrac23\cdot (-3) = \boxed{-\tfrac43}$.

\item Suppose $a_0,a_1,\ldots, a_{2018}$ are integers such that \[(x^2-3x+1)^{1009} = \sum_{k=0}^{2018}a_kx^k\] for all real numbers $x$.  Compute the remainder when $a_0^2 + a_1^2 + \cdots + a_{2018}^2$ is divided by $2017$.

\proposed{David Altizio}

\solution The key to solving this problem is to reason in terms of generating functions.  Recall that for any two polynomials \[p(x) = \sum_{j=0}^kp_jx^j\qquad\text{and}\qquad q(x) = \sum_{j=0}^k q_jx^j\] (where leading coefficients can be zero), their product is given by \[p(x)q(x) = \sum_{j=0}^{2k}\left(\sum_{a+b=j}p_aq_b\right)x^j.\] In terms of this question, note that the polynomial $(x^2-3x+1)^{1009}$ is symmetric\footnote{This follows from the fact that the product of two symmetric polynomials is symmetric: recall that $P$ is symmetric iff $P(x) \equiv x^{\deg P}P(\tfrac 1x)$, and \[(x^{\deg P}P(\tfrac 1x))(x^{\deg Q}Q(\tfrac 1x)) = x^{\deg P + \deg Q}P(\tfrac1x)Q(\tfrac1x).\]}, and so \[a_0^2 + a_1^2 + \cdots + a_{2018}^2 = a_0a_{2018} + a_1a_{2017} + \cdots + a_{2018}a_0.\] This means that in fact the sum is the coefficient of $x^{2018}$ in $\left[(x^2-3x+1)^{1009}\right]^2 = (x^2-3x+1)^{2018}$, which in turn is equal to the constant coefficeint of $(x+\tfrac1x - 3)^{2018}$.  But now the computation is simple, as FLT dictates that \[[x^0]\left(x + \frac 1x - 3\right)^{2018}\equiv [x^0]\left(x^{2017} + \frac{1}{x^{2017}} - 3\right)\left(x + \frac 1x - 3\right)\equiv \boxed 9\pmod{2017}.\]

\item Define a sequence of polynomials $F_n(x)$ by $F_0(x)=0, F_1(x)=x-1$, and for $n\geq 1$, $$F_{n+1}(x)=2xF_n(x)-F_{n-1}(x)+2F_1(x).$$ For each $n$, $F_n(x)$ can be written in the form $$F_n(x)=c_nP_1(x)P_2(x)\cdots P_{g(n)}(x)$$ where $c_n$ is a constant and $P_1(x),P_2(x)\cdots, P_{g(n)}(x)$ are non-constant polynomials with integer coefficients and $g(n)$ is as large as possible. For all $2< n< 101$, let $t$ be the minimum possible value of $g(n)$ in the above expression; for how many $k$ in the specified range is $g(k)=t$?

\proposed{Gunmay Handa}

\solution Firstly, define $T_n(x)=F_n(x)+1$; evidently these polynomials satisfy the recurrence $T_{n+1}(x)=2xT_n(x)-T_{n-1}(x)$. It is not hard to see that $T_n(x)=\cos(n\arccos{x})$ over a suitable interval of convergence. 

\begin{lemma}For $n\geq 1$, 
\[(x-1)[T_{2n+1}(x)-1]=[T_{n+1}(x)-T_n(x)]^2\quad\text{and}\quad 2(x-1)(x+1)[T_{2n}(x)-1]=[T_{n+1}(x)-T_{n-1}(x)]^2.\]\end{lemma}

\begin{proof}This is a direct induction using the recurrence.\end{proof}

\begin{lemma}If $m\mid n$, then $F_{m}(x)\mid F_{n}(x)$. \end{lemma}

\begin{proof}Note that $F_{k}(x)$ has roots where $T_k(x)=1$, or $x=\cos(\frac{2\pi t}{k})$ for $t=0$ through $t=k-1$. Consequently, if $m|n$, then the roots of $F_{m}(x)$ are a subset of the roots of $F_n(x)$, implying the conclusion.\end{proof}

\begin{lemma}If $p$ is an odd prime, then $g(p)=3$.\end{lemma}

\begin{proof}By examining the roots of $F_p(x)$, we see that every root is repeated except for $x=1$; therefore, $F_p(x)=(x-1)G_p(x)^2$ for some polynomial $G_p$. We now claim that $G_p(x)$ is irreducible over $\mathbb{Q}[X]$. To prove this, first note that $T_p(x)=\frac{(x+\sqrt{x^2-1})^p+(x-\sqrt{x^2-1})^p}{2}$ directly from the characteristic polynomial. Substituting $x\mapsto x+1$ and taking $\pmod p$ yields \begin{align*}\frac{T_p(x+1)-1}{x}&\equiv \frac{(x+1+\sqrt{(x+1)^2-1})^p+(x+1-\sqrt{(x+1)^2-1})^p-2}{2x}\\&\equiv\frac{x^p+1^p+x^p+1^p-2}{2x}\equiv x^{p-1} \pmod{p}\end{align*} by the Frobenius endomorphism. Hence, taking the square root yields that all coefficients except that of $x^{\frac{p-1}{2}}$ are divisible by $p$. Now it suffices to check that $p^2$ does not divide the constant term of this polynomial, or $$\lim_{x\rightarrow 1}\frac{\cos(p\arccos{x})-1}{x-1}=\lim_{x\rightarrow 1}\frac{-p\sin(p\arccos x)}{\sqrt{1-x^2}}=p^2$$ by two applications of L'Hopital's rule. Hence, the constant term is $\pm p$, which implies the conclusion by Eisenstein's criterion.\end{proof}

\par Now by Lemma $1$, we know that $g(2n)\geq 4$, so it suffices to only examine odds. Similarly by Lemma $1$, we have that $g(2n+1)\geq 3$, and if $2n+1$ is not prime, we simply apply Lemma $2$ to conclude that the inequality is strict. Hence, the desired count is simply the number of odd primes less than $100$, which is $\boxed{24}$.

\vspace{6pt}

\textbf{Remark.} Lemma 3 is also a direct application of some basic field theory. First notice that the roots of $G_p(x)$ are of the form $\cos(2k\pi/p)$ for $0<k\leq \frac{p-1}{2}$. Let $p$ be an odd prime and consider the tower of field extensions $$\mathbb{Q} \subseteq \mathbb{Q}[\cos(2k\pi/p)] \subseteq \mathbb{Q}[e^{2ik\pi/p}]. $$ Note that $\mathbb{Q}[e^{2ik\pi/p}]$ is of degree $p-1$ over $\mathbb{Q}$, as it precisely corresponds to the $p$th cyclotomic polynomial $\Phi_p(x)=x^{p-1}+x^{p-2}\dots+x+1$. Moreover, from the formula $\cos(2k\pi/p)=(e^{2ik\pi/p}+e^{-2ik\pi/p})/2$, we see that the second field extension is of degree 2 over the first. Finally, by the Tower Law, we see that the roots of $G_p(x)$ are of degree $\frac{p-1}{2}$ and so the polynomial itself must be irreducible. 

\end{enumerate}

\end{document}