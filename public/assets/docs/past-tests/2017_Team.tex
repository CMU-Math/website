\documentclass[10pt]{article}
\usepackage{amsmath, amssymb, amsthm}
\usepackage[top=2cm, left = 2cm, right = 2cm, bottom = 3cm]{geometry}
\usepackage[pdftex]{graphicx}
\usepackage{asymptote}
\usepackage{tikz}
\usetikzlibrary{shapes.geometric, arrows}
\usepackage{fancyhdr}
\newcommand{\N}{\mathbb{N}}
\pagestyle{fancy}
\rhead{}
\chead{\includegraphics[scale=0.17]{CMIMC-header-2017.png}}
\lhead{}
\setlength{\headheight}{43pt}
\rfoot{}
\cfoot{}
\lfoot{}
\newcommand{\proposed}[1]
{
\vspace{5pt}
\noindent\textit{Proposed by #1}
}
\newcommand{\solution}
{
\vspace{5pt}
\noindent\textit{Solution.}\qquad
}
\newcommand{\solcite}[1]
{
\vspace{5pt}
\noindent\textit{Solution (#1).}\qquad
}
\tikzstyle{box} = [rectangle, rounded corners, minimum width=2cm, minimum height=1cm,text centered, draw=black]
\tikzstyle{arrow} = [thick,->,>=stealth]
\newcommand{\calA}{\mathcal{A}}
\newcommand{\calB}{\mathcal{B}}
\begin{document}\thispagestyle{empty}
\begin{center}

\vspace*{90pt}

\includegraphics[scale=0.3]{CMIMC-header-2017.png}

\includegraphics[scale=0.35]{team-header.png}

\vspace{1.6in}

\includegraphics[scale=0.20]{instruction-header.png}
\noindent\rule{17.7cm}{2pt}
\end{center}

\vspace{10pt}

\begin{enumerate}
	\large
	\item Do not look at the test before the proctor starts the round.
	\item This test consists of 10 short-answer problems to be solved in 30
		minutes.
	\item Write your team name and team ID on your answer sheet.
	\item Write your answers in the corresponding boxes on the answer
		sheets.
	\item No computational aids other than pencil/pen are permitted.
	\item All answers are integers.
	\item If you believe that the test contains an error, submit your
		protest in writing to Porter 100.
\end{enumerate}

\newpage

\begin{center}
\Large\textbf{Team}\normalsize
\end{center}

\begin{enumerate}

\item Find the integer $n$ such that
\[n + \left\lfloor\sqrt{n}\right\rfloor + \left\lfloor\sqrt{\sqrt{n}}\right\rfloor = 2017.\] Here, as usual, $\lfloor\cdot\rfloor$ denotes the floor function.

\item Suppose $x$, $y$, and $z$ are nonzero complex numbers such that $(x+y+z)(x^2+y^2+z^2)=x^3+y^3+z^3$.  Compute \[(x+y+z)\left(\dfrac1x+\dfrac1y+\dfrac1z\right).\]

\item Suppose Pat and Rick are playing a game in which they take turns writing numbers from $\{1, 2, \dots, 97\}$ on a blackboard. In each round, Pat writes a number, then Rick writes a number; Rick wins if the sum of all the numbers written on the blackboard after $n$ rounds is divisible by 100. Find the minimum positive value of $n$ for which Rick has a winning strategy.

\item Say an odd positive integer $n > 1$ is \textit{twinning} if $p - 2 \mid n$ for every prime $p \mid n$. Find the number of twinning integers less than 250.

\item We have four registers, $R_1,R_2,R_3,R_4$, such that $R_i$ initially contains the number $i$ for $1\le i\le4$. We are allowed two operations:

\begin{itemize}
\item Simultaneously swap the contents of $R_1$ and $R_3$ as well as $R_2$ and $R_4$.

\item Simultaneously transfer the contents of $R_2$ to $R_3$, the contents of $R_3$ to $R_4$, and the contents of $R_4$ to $R_2$. (For example if we do this once then $(R_1,R_2,R_3,R_4)=(1,4,2,3)$.)
\end{itemize}

Using these two operations as many times as desired and in whatever order, what is the total number of possible outcomes?

\item George is taking a ten-question true-false exam, where the answer key has been selected uniformly at random; however, he doesn't know any of the answers! Luckily, a friend has helpfully hinted that no two consecutive questions have true as the correct answer. If George takes the exam and maximizes the expected number of questions he gets correct, how many of his answers are expected to be right?

\item Define $\{p_n\}_{n=0}^\infty\subset\N$ and $\{q_n\}_{n=0}^\infty\subset\N$ to be sequences of natural numbers as follows:

\begin{itemize}

\item $p_0=q_0=1$;

\item For all $n\in\N$, $q_n$ is the smallest natural number such that there exists a natural number $p_n$ with $\gcd(p_n,q_n)=1$ satisfying \[\dfrac{p_{n-1}}{q_{n-1}} < \dfrac{p_n}{q_n} < \sqrt 2.\]

\end{itemize}

Find $q_3$.

\item Alice and Bob have a fair coin with sides labeled $C$ and $M$, and they flip the coin repeatedly while recording the outcomes; for example, if they flip two $C$'s then an $M$, they have $CCM$ recorded. They play the following game: Alice chooses a four-character string $\calA$, then Bob chooses two distinct three-character strings $\calB_1$ and $\calB_2$ such that neither is a substring of $\calA$. Bob wins if $\calA$ shows up in the running record before either $\calB_1$ or $\calB_2$ do, and otherwise Alice wins. Given that Alice chooses $\calA = CMMC$ and Bob plays optimally, compute the probability that Bob wins.

\item Circles $\omega_1$ and $\omega_2$ are externally tangent to each other.  Circle $\Omega$ is placed such that $\omega_1$ is internally tangent to $\Omega$ at $X$ while $\omega_2$ is internally tangent to $\Omega$ at $Y$.  Line $\ell$ is tangent to $\omega_1$ at $P$ and $\omega_2$ at $Q$ and furthermore intersects $\Omega$ at points $A$ and $B$ with $AP<AQ$.  Suppose that $AP=2$, $PQ=4$, and $QB=3$.  Compute the length of line segment $\overline{XY}$.

\item The polynomial $P(x) = x^3 - 6x - 2$ has three real roots, $\alpha$, $\beta$, and $\gamma$.  Depending on the assignment of the roots, there exist two different quadratics $Q$ such that the graph of $y=Q(x)$ pass through the points $(\alpha,\beta)$, $(\beta,\gamma)$, and $(\gamma,\alpha)$.  What is the larger of the two values of $Q(1)$?

\end{enumerate}

\end{document}
