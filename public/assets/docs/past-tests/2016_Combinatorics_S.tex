\documentclass[10pt]{article}
\usepackage{amsmath, amssymb, amsthm}
\usepackage[top=2cm, left = 2cm, right = 2cm, bottom = 3cm]{geometry}
\usepackage[pdftex]{graphicx}
\usepackage{asymptote}
\usepackage{fancyhdr}
\pagestyle{fancy}
\rhead{}
\chead{\includegraphics[scale=0.12]{CMIMC-header.png}}
\lhead{}
\setlength{\headheight}{43pt}
\rfoot{}
\cfoot{}
\lfoot{}
\newcommand{\proposed}[1]
{
\vspace{5pt}
\noindent\textit{Proposed by #1}
}
\newcommand{\solution}
{
\vspace{5pt}
\noindent\textit{Solution.}\qquad
}
\newcommand{\comment}[1]{}
\begin{document}

\begin{center}
\huge\textbf{Combinatorics Solutions}\normalsize

\vspace{3pt}
\end{center}

\begin{enumerate}

\item The phrase ``COLORFUL TARTAN'' is spelled out with wooden blocks, where blocks of the same letter are indistinguishable. How many ways are there to distribute the blocks among two bags of different color such that neither bag contains more than one of the same letter?

\proposed{Joshua Siktar}

\solution Observe that there are five pairs of letters and four singletons. It is not necessary to care about the pairs, since each pair must have one letter in each bag. It then remains to distribute four distinct letters among two distinguishable bags; this can be done in $2^4 = \boxed{16}$ ways.

\comment{
\item Xinyu writes a computer program that randomly rearranges the digits 0, 2, 4, 6, and 8 to create a five-digit number with no leading zeroes. If she executes this program once, the probability the program outputs an integer divisible by 4 can be written in the form $\tfrac{m}{n}$ where $m$ and $n$ are positive integers which share no common factors. What is $m+n$?

\proposed{David Altizio}

\solution Note that Xinyu's integer is divisible by four if and only if the last two digits of her integer together form a two-digit number which is divisible by 4. As a result, Xinyu's integer must end in one of 04, 08, 20, 24, 28, 40, 44, 48, 60, 64, 68, 80, 84, or 88. Now we perform casework based on whether a 0 appears in the last two digits.
\begin{itemize}
\item Case 1: There is a zero. Then the first three digits can be ordered in any possible way, leading to a total of $3! = 6$ possibilities. Since six of the above endings have a 0, there are $6 \cdot 6 = 36$ possibilities for this case.
\item Case 2: There is no zero. In this case, there are only four possibilities for the first three digits (X0Y, Y0X, XY0, YX0). Since there are eight possibilities for the last two digits for which such a case may exist, there are $8\cdot 4 = 32$ total outcomes here.
\end{itemize}
All in all, there are $36+32=68$ numbers which satisfy the condition. Since the total number of outputs Xinyu's program can print is $5!−4!=96$, the desired probability is $\tfrac{68}{96}=\tfrac{17}{24}$, and the answer is thus $\boxed{41}$.}

\item Six people each flip a fair coin. Everyone who flipped tails then flips their coin again. Given that the probability that all the coins are now heads can be expressed as simplified fraction $\tfrac{m}{n}$, compute $m+n$.

\proposed{Patrick Lin}

\solution Observe that each person has a $1 - (\tfrac12)(\tfrac12) = \tfrac34$ chance of ending up with a head; the only way they do not end heads is if they flip tails twice in a row. Hence with 6 people the probability is $(\tfrac34)^6 = \tfrac{729}{4096}$, and so the answer is $\boxed{4825}$.

\item At CMU, markers come in two colors: blue and orange. Zachary fills a hat randomly with three markers such that each color is chosen with equal probability, then Chase shuffles an additional orange marker into the hat. If Zachary chooses one of the markers in the hat at random and it turns out to be orange, the probability that there is a second orange marker in the hat can be expressed as simplified fraction $\tfrac{m}{n}$. Compute $m+n$.

\proposed{Patrick Lin}

\solution Notice that there is a $\tfrac18$, $\tfrac38$, $\tfrac38$, and $\tfrac18$ chance for the hat to contain 1, 2, 3, and 4 orange markers, respectively, since the original three markers are random and we add one extra orange marker in. Then the probability that we choose an orange marker first is $\tfrac18 \cdot \tfrac14 + \tfrac38 \cdot \tfrac24 + \tfrac38 \cdot \tfrac34 + \tfrac18 \cdot \tfrac44 = \tfrac{20}{32}$, and the probability that there is no orange marker left given this is equal to the chance there was only one orange marker given this, which is $\tfrac18 \cdot \tfrac14 = \tfrac1{32}$. Hence the chance there is another orange marker is given by $1 - \tfrac{1/32}{20/32} = \tfrac{19}{20}$, and the answer is thus $\boxed{39}$.

\comment{
\item Kevin is located on the top face of a regular dodecahedron. Every minute, he moves randomly to an adjacent face. Determine the expected amount of time, in minutes, that will pass before he moves to the bottom face for the first time.

\proposed{Patrick Lin}

\solution Let $E_n$ represent the expected amount of time it takes Kevin to move to the bottom face, where $n$ is the minimum distance, in time steps, from the bottom. Then clearly we want to find $E_3$. By examining the geometry of a dodecahedron, we have the following recurrences:
\begin{align*}
E_0 & = 0 \\
E_1 & = 1 + \tfrac15 E_0 + \tfrac25 E_1 + \tfrac25 E_2 \\
E_2 & = 1 + \tfrac25 E_1 + \tfrac25 E_2 + \tfrac15 E_3 \\
E_3 & = 1 + E_2
\end{align*}
Solving this yields $E_3 = \boxed{15}$.}

\item Kevin colors three distinct squares in a 3x3 grid. Given that there exist two uncolored squares such that coloring either one of them would create a horizontal or vertical red line, find the number of ways he could have colored the original three squares.

\proposed{Patrick Lin}

\solution Observe that, in the original three colored squares, one pair must share the same row and another pair must share the same column in order to guarantee being able to create two different lines. Assume we pick the first square arbitrarily from 9 choices. If we pick the second square sharing the same row or column as the first, we have 4 options, and then we have 4 options for the third. If we pick the second square not sharing either the same row or the same column as the first, we still have 4 options, but then we have only 2 options for the third. This overcounts by a factor of $3! = 6$, and so in total there are $\tfrac16 (9\cdot4\cdot4 + 9\cdot4\cdot2) = \boxed{36}$ ways.

\item Let $\mathcal{S}$ be a regular 18-gon, and for two vertices in $\mathcal{S}$ define the \textit{distance} between them to be the length of the shortest path along the edges of $mathcal{S}$ between them (e.g. adjacent vertices have distance 1). Find the number of ways to choose three distinct vertices from $\mathcal{S}$ such that no two of them have distance 1, 8, or 9.

\proposed{Patrick Lin}

\solution Consider the nine pairs of vertices formed by pairing a vertex with its diametrically opposite vertex. Clearly, no pair can have both its vertices chosen, since they have distance 9. Further, choosing one from the pair is equivalent to choosing the other - a vertex with distance 1 from one will have distance 8 from the other, and vice versa - and hence we can consider the two vertices identical. The problem then reduces to finding the number of ways to choose three distinct vertices in a nonagon, where each vertex represents one such pair in $\mathcal{S}$, such that no two are pairwise adjacent. Choosing the first from 9 vertices, we find there are $\binom62 - 5 = 10$ ways to choose the remaining two. But we have overcounted by distinguishing the first vertex, and hence there are actually $9\cdot10\cdot\frac13 = 30$ ways. Now for each choice of three pairs there are $2^3 = 8$ ways to choose the individual vertices, and thus the answer is $8\cdot30 = \boxed{240}$.

\par \textit{Thanks to Brice Huang for suggesting an alteration to this problem.}

\item Shen, Ling, and Ru each place four slips of paper with their name on it into a bucket. They then play the following game: slips are removed one at a time, and whoever has all of their slips removed first wins. Shen cheats, however, and adds an extra slip of paper into the bucket, and will win when four of his are drawn. Given that the probability that Shen wins can be expressed as simplified fraction $\tfrac{m}{n}$, compute $m+n$.

\proposed{Victor Xu, solution by Patrick Lin}

\solution First, observe that the probability that Ling and Ru win (or lose) are equal. Then
\[2P(\text{L loses}) = P(\text{L loses}) + P(\text{R loses}) = 2P(\text{S wins}) + P(\text{L wins}) + P(\text{R wins}) = 1 + P(\text{S wins}).\]
It then suffices to compute the probability that Ling loses. Consider continuing to draw slips until all the slips have been drawn. Then if Ling’s slip is the last one, clearly she loses. If Ru’s slip is the last one, then Ling loses if the sequence with all instances of $R$ removed ends either with $L$ or with $LS$. If Shen’s slip is the last one removed, then the game is entirely symmetric and so everyone has an equal chance to lose. We count probability in each case.
\begin{itemize}
\item Case 1: L last. Given this, she loses with probability 1.
\item Case 2: R last. Given this, the probability she loses is given by $\tfrac49+\tfrac59\tfrac48$.
\item Case 3: S last. Given this, everyone has an equal chance to lose and so her probability of losing is $\tfrac23$.
\end{itemize}
Thus the overall probability she loses is
\[\frac{4}{13}+\frac{4}{13}\left(\frac49+\frac59\cdot\frac48\right) + \frac{5}{13}\cdot\frac23 = \frac{92}{117}.\]
Hence the desired probability is equal to $2\cdot\tfrac{92}{117} - 1 = \tfrac{67}{117}$, and the answer is $\boxed{184}$.

\item There are eight people, each with their own horse. The horses are arbitrarily arranged in a line from left to right, while the people are lined up in random order to the left of all the horses. One at a time, each person moves rightwards in an attempt to reach their horse. If they encounter a mounted horse on their way to their horse, the mounted horse shouts angrily at the person, who then scurries home immediately. Otherwise, they get to their horse safely and mount it. The expected number of people who have scurried home after all eight people have attempted to reach their horse can be expressed as simplified fraction $\tfrac{m}{n}$. Find $m+n$.

\proposed{Patrick Lin}

\solution We find the expected number of people who mount their horse successfully. Number the horses from left to right $1, 2, \dots, 8$ and label their owners with the same number. Note that when person $i$ moves, the only way for him to mount his horse is if he is the first of $i$ people (namely, those with labels $1, 2, \dots, i$) to move. Hence person $i$ mounts his horse with $\tfrac{1}{i}$ probability. Since we have eight people, the expected number of people who mount their horse is thus $\sum_{i=1}^8 \tfrac{1}{i} = H_8 = \tfrac{761}{280}$. The expected number of people who scurry home is hence
\[8 - \frac{761}{280} = \frac{1479}{280},\]
and so the answer is $\boxed{1759}$.

\item Brice is eating bowls of rice. He takes a random amount of time $t_1 \in (0,1)$ minutes to consume his first bowl, and every bowl thereafter takes $t_n = t_{n-1} + r_n$ minutes, where $t_{n-1}$ is the time it took him to eat his previous bowl and $r_n \in (0,1)$ is chosen uniformly and randomly. The probability that it takes Brice at least 12 minutes to eat 5 bowls of rice can be expressed as simplified fraction $\tfrac{m}{n}$. Compute $m+n$.

\proposed{Patrick Lin}

\solution We consider geometric probability in five dimensions. Note that the probability that it takes Brice at least 12 minutes is equal to the probability that he takes at most 3 minutes. Observe that the volume for the figure bounded by $x_1 + x_2 + x_3 + x_4 + x_5 \le 3$ and $x_i > 0$ is a triangular hyperpyramid with side length 3 whose volume is given by $\frac{3^5}{5!}$. But for each $i$, we have the bound $x_i \le i$, which forms a box with volume 5!. Now observe that since $3 = 1 + 2$, the volume of the figure outside the box is equal to a similar hyperpyramid with side length 2 (this comes when $x_1 > 1$) and another one with side length 1 (this comes when $x_2 > 2$). There is no other overlap, since it is impossible for both $x_1$ and $x_2$ to exceed their respective bounds, and it is impossible for each of the other $x_i$’s to exceed their bounds as well. Hence the total volume is given by $\frac{3^5 - 2^5 - 1^5}{5!}$. The probability is thus $\frac{3^5 - 2^5 - 1^5}{(5!)^2} = \frac{7}{480}$, and so the answer is $\boxed{487}$.

\item 1007 distinct potatoes are chosen independently and randomly from a box of 2016 potatoes numbered $1, 2, \dots, 2016$, with $p$ being the smallest chosen potato. Then, potatoes are drawn one at a time from the remaining 1009 until the first one with value $q < p$ is drawn. If no such $q$ exists, let $S = 1$. Otherwise, let $S = pq$. Then given that the expected value of $S$ can be expressed as simplified fraction $\tfrac{m}{n}$, compute $m+n$.

\proposed{Patrick Lin}

\solution Note that the largest possible value for $p$ is equal to $(2016 - 1007) + 1 = 1010$. Then for any $1 \le i \le 1010$, the probability that $p = i$ is given by
\[P(p = i) = \frac{\binom{2016-i}{1006}}{\binom{2016}{1007}}.\]
Hence for $i > 1$, $q$ is chosen randomly between 1 and $i - 1$, inclusive, and so $E[q \mid p = i] = \tfrac{p}{2}$. This means $E[S \mid p = i] = \tfrac{p^2}{2}$. Note that for $i = 1$, we have $E[S \mid p = 1] = \tfrac{1}{2}$, which abides by the observation for $i > 1$, and so we can just lump everything together. The total expected value is hence
\[E[S] = \sum_{i=1}^{1010} \frac{\binom{2016-i}{1006}}{\binom{2016}{1007}}\cdot\frac{i^2}{2} = \frac{1}{2\binom{2016}{1007}} \sum_{i=1}^{1010} i^2 \binom{2016-i}{1006}.\]
With some application of Hockey Stick, we find that
\begin{align*}
2\binom{2016}{1007} E[S] & = 1^2 \binom{2015}{1006} + 2^2 \binom{2014}{1006} + \dots + 1010^2 \binom{1006}{1006} \\
& = 1 \binom{2016}{1007} + 3 \binom{2015}{1007} + \dots + 2019 \binom{1007}{1007} \\
& = 2\left(1 \binom{2016}{1007} + 2 \binom{2015}{1007} + \dots + 1010 \binom{1007}{1007}\right) - \binom{2017}{1008} \\
& = 2\left(\binom{2017}{1008} + \binom{2016}{1008} + \dots + \binom{1008}{1008}\right) - \binom{2017}{1008} \\
& = 2 \binom{2018}{1009} - \binom{2017}{1008} \\
& = \binom{2018}{1009} + \binom{2017}{1009}.
\end{align*}
Hence we have
\begin{align*}
E[S] & = \frac{\binom{2018}{1009} + \binom{2017}{1009}}{2\binom{2016}{1007}} = \frac{\frac{2018!}{1009!1009!} + \frac{2017!}{1009!1008!}}{2\frac{2016!}{1009!1007!}} \\
& = \frac{3\cdot 1009 \cdot 2017!}{2\cdot 1008\cdot 1009\cdot 2016!} = \frac{2017}{672},
\end{align*}
and so the answer is $\boxed{2689}$.

\item For all positive integers $m\geq 1$, denote by $\mathcal{G}_m$ the set of simple graphs with exactly $m$ edges.  Find the number of pairs of integers $(m,n)$ with $1<2n\leq m\leq 100$ such that there exists a simple graph $G\in\mathcal{G}_m$ satisfying the following property: it is possible to label the edges of $G$ with labels $E_1$, $E_2$, $\ldots$, $E_m$ such that for all $i\neq j$, edges $E_i$ and $E_j$ are adjacent if and only if either $|i-j|\leq n$ or $|i-j|\geq m-n$.

\par \textit{Note: }A graph is said to be \textit{simple} if it has no self-loops or multiple edges. In other words, no edge connects a vertex to itself, and the number of edges connecting two distinct vertices is either $0$ or $1$.

\proposed{David Altizio}

\solution For convenience, we make a few definitions:

\begin{itemize}

\item Let $f$ be a function which takes in a graph $G=(V,E)$ and returns another graph $G'=(V',E')$ such that there exists a bijection $g:V'\mapsto E$ with the property that the edge $\{v_1,v_2\}$ is in $E'$ if and only if $g(v_1)$ and $g(v_2)$ are both incident to some common vertex $v\in V$.

\item For positive integers $m$ and $n$ with $m\geq2n$, let $C_{m,n}$ denote the graph with vertex sequence $\{v_i\}_{i=1}^m$ such that vertices $v_i$ and $v_j$ are adjacent iff $|i-j|\leq n$ or $|i-j|\geq m-n$.

\end{itemize} 

Note that the problem statement is equivalent to finding the number of pairs of integers $(m,n)$ such that there exists a graph $H$ with $f(H)=C_{m,n}$.\footnote{This was the original formulation of the problem; as such, the problem author has decided to re-introduce this notation into the solution in order to minimize as much re-typing as possible.} With that in mind, we claim that there are only three possible classes of pairs $(m,n)$ for which an $H$ exists:

\begin{itemize}

\item $(m,n)=(i,1)$ for $2\leq i\leq 100$;

\item $(m,n)=(j,\lfloor\tfrac j2\rfloor)$ for $4\leq j\leq 100$ (note that $j=2$ and $j=3$ are already accounted for);

\item $(m,n)=(6,2)$.

\end{itemize}

This yields $99+97+1=\boxed{197}$ possible pairs.

\par To prove this, we case on the value of $n$.  The cases are ordered by difficulty.

\begin{itemize}

\item \textbf{CASE 1: }$n=1$.  First consider the case $(m,n)=(2,1)$.  Let $G$ be a path of length $2$.  It is not hard to show that $f(G)=C_{2,1}$.  Hence $m=2$ works.  Otherwise, note that if $G$ is a cycle of length $k$, then $f(G)$ is also a cycle of length $k$.  (Why?) Hence all cycles of length $k\geq 3$ work, and these are only achieved by $n=1$.  (Note that $C_{3,2}$ is also a cycle of length $3$, but this is disallowed by the condition $2n\leq m$.)

\item \textbf{CASE 2: }$n\geq 3$.  I claim that the only conditions that work in this case are cliques.  To prove this, we first make an important observation.  Assume that $C_{m,n}$ has a clique of size $k>3$.  Then note that all vertices in this clique are connected to each other, meaning that the edges in $H$ associated with these vertices must all touch each other.  The only way this can happen is when all these edges are incident to some common vertex.  (The case $k=3$ is special, as we will discuss later; this is the reason why $n=2$ is a separate case.)  Running the reverse logic, it is not hard to show that $f(G)$ has a clique of size at least $k$ iff $G$ has a vertex of degree at least $k$. 

\par Suppose that $m>2n$, i.e. $C_{m,n}$ is not an $m$-clique.  Consider the vertex $v_1$.  Remark that $v_{n+2}$ and $v_1$ are not connected, and furthermore note that $v_2$, $v_3$, $\ldots$, $v_{n+1}$ are all connected to both of these vertices and to each other.  In other words, $v_1$ through $v_{n+1}$ form a clique, as do $v_2$ through $v_{n+2}$.  Thus, all edges associated with $v_1$ through $v_{n+2}$ must be incident to a common vertex, but this is a contradiction since $v_1$ and $v_{n+2}$ are not adjacent!  Hence $C_{m,n}$ must be an $m$-clique, which forces $(m,n)=(k,\lfloor\tfrac k2\rfloor)$.  Note that this is constructable for all $k$ by considering a graph $G$ with $k$ edges all incident at a single vertex.

\item \textbf{CASE 3: }$n=2$.  In order to tackle this case, we need to explicitly construct a connected subgraph of $H$ in order to derive the contradictions and examples we need.

\par Assume $m\geq 7$.  Consider vertex $v_1$, and for convenience let $e_i\equiv g(v_i)$ for all $1\leq i\leq m$.  Note that the edge $e_1$ is incident to exactly four other edges: $e_2$, $e_3$, $e_m$, and $e_{m-1}$.  I now claim that although $e_2$ and $e_m$ are connected, they cannot be incident to the same vertex of $e_1$.  To prove this, write $e_1=\{A,B\}$, and assume WLOG that $e_2$ and $e_m$ are incident to $A$.  Note that $e_3$ and $e_{m-1}$ are not connected by the definition of $C_{m,2}$.  Thus, these two edges must not both be adjacent to $B$, meaning that $A$ has degree at least $4$.  But this is a contradiction, since by the logic in Case 2 $C_{m,2}$ must have a clique of size $\geq 4$, which is false.  Hence $e_2$ and $e_m$ are incident to opposite endpoints of $e_1$.  Combined with the fact that $e_3$ and $e_{m-1}$ are not adjacent, we can conclude that the graph must be of the following form:

\begin{center}\includegraphics[scale=0.5]{CMIMCComboGraph.PNG}\end{center}

Now it is not hard to derive a contradiction.  Consider the edge $g_4$.  Remark that $g_4$ must be connected to both $g_3$ and $g_2$.  However, if these three edges were to all share a common vertex, then $g_4$ would be connected to $g_1$, which is impossible.  Thus, the only placement for $g_4$ occurs when $g_2$, $g_3$, and $g_4$ form a triangle.  But this means that $g_4$ is connected to $g_m$, contradiction!  Hence $m\geq 7$ is impossible.

\par For $m=4$ and $m=5$, the construction from Case 2 works, while $m=6$ works by noting that $f(K_4)=C_{6,2}$.  Hence $(4,2)$, $(5,2)$, and $(6,2)$ work as well.

\end{itemize}

Combining all these cases, we get the three families of solutions listed at the beginning of this solution, and so we are done.

\end{enumerate}

\end{document}
