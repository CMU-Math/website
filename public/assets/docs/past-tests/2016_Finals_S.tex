\documentclass[10pt]{article}
\usepackage{amsmath, amssymb, amsthm}
\usepackage[top=2cm, left = 2cm, right = 2cm, bottom = 3cm]{geometry}
\usepackage[pdftex]{graphicx}
\usepackage{asymptote}
\usepackage{tikz}
\usepackage{fancyhdr}
\pagestyle{fancy}
\rhead{}
\chead{\includegraphics[scale=0.2]{CMIMC-header.png}}
\lhead{}
\setlength{\headheight}{43pt}
\rfoot{}
\cfoot{}
\lfoot{}
\newcommand{\proposed}[1]
{
\vspace{5pt}
\noindent\textit{Proposed by #1}
}
\newcommand{\solution}
{
\vspace{5pt}
\noindent\textit{Solution.}\qquad
}
\newcommand{\comment}[1]{}
\begin{document}

\begin{center}
\huge\textbf{Algebra Tiebreaker Solutions}\normalsize

\vspace{3pt}
\end{center}

\begin{enumerate}

\item Let \[f(x)=\dfrac{1}{1-\dfrac{1}{1-x}}\,.\] Compute $f^{2016}(2016)$, where $f$ is composed upon itself $2016$ times.

\proposed{Joshua Siktar}

\solution We compute $f(2016)=\frac{2015}{2016}$, $f(\frac{2015}{2016})=-\frac1{2015}$, and $f(-\frac1{2015})=2016$. Therefore, $f$ is periodic with period $3$, and so $f^{2016}(2016)=f^{2016\pmod3}(2016)=f^{3}(2016)=\boxed{2016}$.

\item Determine the value of the sum \[\left|\sum_{1\leq i<j\leq 50}ij(-1)^{i+j}\right|.\]

\proposed{David Altizio}

\solution Let $a_i=i(-1)^i$. Then since \[\left(\sum_{1\le i\le50}a_i\right)^2=\sum_{1\le i,j\le50}a_ia_j=2\sum_{1\le i<j\le50}a_ia_j+\sum_{1\le i\le50}a_i^2,\] we have \[2\sum_{1\le i<j\le50}a_ia_j=(-1+2-3+4+\dots+50)^2-(1^2+\dots+50^2)=25^2-\frac1650(51)(101)=-42300,\] whence the answer is $\boxed{21150}$.

\item Suppose $x$ and $y$ are real numbers which satisfy the system of equations \[x^2-3y^2=\frac{17}x\qquad\text{and}\qquad 3x^2-y^2=\frac{23}y.\] Then $x^2+y^2$ can be written in the form $\sqrt[m]{n}$, where $m$ and $n$ are positive integers and $m$ is as small as possible.  Find $m+n$.

\proposed{David Altizio}

\solution Note that the equations rearrange to $x^3-3xy^2=17$ and $3x^2y-y^3=23$.  Thus \[x^3-3xy^2+i(3x^2y-y^3)=17+23i\implies (x+yi)^3=17+23i.\] Taking the magnitude of both sides yields \[(x^2+y^2)^{3/2}=(17+23i)^{1/2}\implies x^2+y^2=\sqrt[3]{17^2+23^2}=\sqrt[3]{818}.\] The requested answer is $818+3=\boxed{821}$. 

\end{enumerate}

\newpage

\begin{center}
\huge\textbf{Combinatorics Tiebreaker}\normalsize

\vspace{3pt}
\end{center}

\begin{enumerate}

\item For a set $S \subseteq \mathbb{N}$, define $f(S) = \{\left\lceil \sqrt{s} \right\rceil \mid s \in S\}$. Find the number of sets $T$ such that $\vert f(T) \vert = 2$ and $f(f(T)) = \{2\}$. %ID: 65; patrick

\proposed{Patrick Lin}

\solution Denote $S_n = \{k \mid \lceil k \rceil = n\}$, and $a_n$ the number of non-empty subsets of $S_n$. Observe that $S_n$ contains exactly $2n - 1$ elements, and so $a_n$ contains $2^{2n-1} - 1$ elements. Since $f^2(T) = \{2\}$, it follows that $f(T) \subset S_2 = \{2,3,4\}$; in particular, $f(T)$ contains exactly two of those three elements. Thus the number of sets $T$ that satisfy the problem condition is given by $a_2a_3 + a_2a_4 + a_3a_4 = (7)(31) + (7)(127) + (31)(127) = \boxed{5043}$.

\item Let $S = \{1,2,3,4,5,6,7\}$. Compute the number of sets of subsets $T = \{A, B, C\}$ with $A, B, C \in S$ such that $A \cup B \cup C = S$, $(A \cap C) \cup (B \cap C) = \emptyset$, and no subset contains two consecutive integers. %ID: 17; patrick

\proposed{Patrick Lin}

\solution The last condition is equivalent to $(A \cup B) \cap C = \emptyset$, and $A \cup B \cup C = S$ tells us each element is in at least one set. Hence each element has four possible states: in $A$ only, in $B$ only, in both $A$ and $B$, and in $C$ only. Letting $A_n$ be the number of sets $T$ that satisfy the condition for $S = \{1,2,\dots,n\}$ such that $n$ is in only $A$ and similarly for $B$, $AB$, and $C$, we have
\begin{align*}
A_{n+1} & = B_n + C_n \\
B_{n+1} & = A_n + C_n \\
AB_{n+1} & = C_n \\
C_{n+1} & = A_n + B_n + AB_n,
\end{align*}
with $A_1 = B_1 = AB_1 = C_1 = 1$. Solving this up to $n=7$ yields $A_7 + B_7 + AB_7 + C_7 = \boxed{394}$.

\item Let $S$ be the set containing all positive integers whose decimal representations contain only 3�s and 7�s, have at most 1998 digits, and have at least one digit appear exactly 999 times. If $N$ denotes the number of elements in $S$, find the remainder when $N$ is divided by 1000. %ID: 40; patrick

\proposed{Patrick Lin}

\solution Note that element in $S$ contains at least 999 and at most 1998 digits. For some number $0 \le k < 999$, the number of integers of length $999+k$ in $S$ is equal to $2\binom{999+k}{999}$, and for $k=999$ there are $\binom{1998}{999}$ elements in $S$. Hence $N = 2\sum_{k=0}^{999} \binom{999+k}{999} - \binom{1998}{999} = 2\binom{1999}{999} - \binom{1998}{999} = \binom{2000}{1000} - \binom{1998}{999}$. By examining $v_2(2000!)$ and $v_2(1000!)$, we find that $\binom{2000}{1000} \equiv 0 \pmod 8$, and using Wolstenholme's yields $\binom{2000}{1000} \equiv \binom{16}{8} \equiv 120 \pmod{125}$. Hence $\binom{2000}{1000} \equiv 120 \pmod{1000}$. Noting that $\binom{1998}{999} = \tfrac{500}{1999} \binom{2000}{1000}$ yields $\binom{1998}{999} \equiv 0 \pmod{1000}$, and so the answer is $120 - 0 = \boxed{120}$.

\end{enumerate}

\newpage

\begin{center}
\huge\textbf{Computer Science Tiebreaker}\normalsize

\vspace{3pt}
\end{center}

\begin{enumerate}

\item A \emph{planar} graph is a connected graph that can be drawn on a sphere without edge crossings. Such a drawing will divide the sphere into a number of faces. Let $G$ be a planar graph with $11$ vertices of degree $2$, $5$ vertices of degree $3$, and $1$ vertex of degree $7$. Find the number of faces into which $G$ divides the sphere.

\proposed{Cody Johnson}

\solution By double counting, the total number of edges is $\frac12(11\cdot2+5\cdot3+1\cdot7)=22$. Thus, by Euler's polyhedron formula, there are $F=E-V+2=22-17+2=\boxed{7}$ faces. Note: there does exist such a graph, for example:

\begin{center}
\begin{tikzpicture}
\draw [fill=black] (0,0) circle (0.05);
\draw [fill=black] (0.5,0) circle (0.05);
\draw [fill=black] (1,0) circle (0.05);
\draw [fill=black] (1.5,0) circle (0.05);
\draw [fill=black] (2,0) circle (0.05);
\draw [fill=black] (2.5,0) circle (0.05);
\draw [fill=black] (3,0) circle (0.05);
\draw [fill=black] (3.5,0) circle (0.05);
\draw [fill=black] (4,0) circle (0.05);
\draw [fill=black] (4.5,0) circle (0.05);
\draw [fill=black] (5,0) circle (0.05);
\draw [fill=black] (5.5,0) circle (0.05);
\draw [fill=black] (6,0) circle (0.05);
\draw [fill=black] (6.5,0) circle (0.05);
\draw [fill=black] (7,0) circle (0.05);
\draw [fill=black] (7.5,0) circle (0.05);
\draw [fill=black] (1.25,2) circle (0.05);

\draw (0,0) -- (7.5,0);
\foreach \x in {0,1,2,3,4,5,15}
\draw (\x/2,0) -- (1.25,2);
\end{tikzpicture}
\end{center}

\item The \emph{Stooge sort} is a particularly inefficient recursive sorting algorithm defined as follows: given an array $A$ of size $n$, we swap the first and last elements if they are out of order; we then (if $n\ge3$) Stooge sort the first $\lceil\frac{2n}3\rceil$ elements, then the last $\lceil\frac{2n}3\rceil$, then the first $\lceil\frac{2n}3\rceil$ elements again. Given that this runs in $O(n^\alpha)$, where $\alpha$ is minimal, find the value of $(243/32)^\alpha$.

\proposed{Cody Johnson}

\solution Let $T(n)$ be the number of steps Stooge sort takes on an array of size $n$. Then \[T(n)=3T\left(\frac23n\right)=3^2T\left(\left(\frac23\right)^2n\right)=3^3T\left(\left(\frac23\right)^3n\right)=\dots.\] Let $k$ be the smallest integer such that $(\frac23)^kn\le1$. Then $k\ge\log_{3/2}n$, so the algorithm will run in $O(3^{\log_{3/2}n})=O(n^{\log_{3/2}3})$. Finally, $(243/32)^{\log_{3/2}3}=(3/2)^{\log_{3/2}3^5}=3^5=\boxed{243}$.

\item Let $\varepsilon$ denote the empty string. Given a pair of strings $(A,B)\in\{0,1,2\}^*\times\{0,1\}^*$, we are allowed the following operations:

\[\begin{cases}
(A,1)\to(A0,\varepsilon)\\
(A,10)\to(A00,\varepsilon)\\
(A,0B)\to(A0,B)\\
(A,11B)\to(A01,B)\\
(A,100B)\to(A0012,1B)\\
(A,101B)\to(A00122,10B)
\end{cases}\]

We perform these operations on $(A,B)$ until we can no longer perform any of them. We then iteratively delete any instance of $20$ in $A$ and replace any instance of $21$ with $1$ until there are no such substrings remaining. Among all binary strings $X$ of size $9$, how many different possible outcomes are there for this process performed on $(\varepsilon,X)$?

\proposed{Cody Johnson}

\solution Let $[\cdot]$ denote the value when we read $\cdot$ as a binary integer. Now we claim this process performed on $(\varepsilon,X)$ will output $\lfloor[X]/3\rfloor$ (with enough leading zeroes). It is clear for small enough values of $|X|$. Now consider the following algorithm for division when the first digits are $100$ or $101$: %%%%%%%%%%%%%%%%%%%%%% incomplete

\end{enumerate}

\newpage

\begin{center}
\huge\textbf{Geometry Tiebreaker}\normalsize

\vspace{3pt}
\end{center}

\begin{enumerate}

\item Point $A$ lies on the circumference of a circle $\Omega$ with radius $78$.  Point $B$ is placed such that $AB$ is tangent to the circle and $AB=65$, while point $C$ is located on $\Omega$ such that $BC=25$.  Compute the length of $\overline{AC}$. %David Altizio

\proposed{David Altizio}

\solution Extend $PC$ past $C$ to intersect $\Omega$ at $D$.  Then by Power of a Point $AB^2=AC\cdot AD$, so \[AD=\tfrac{AB^2}{AC}=\dfrac{65^2}{25}=169.\] Now let $D'$ be the point on $\Omega$ such that $AD'$ is a diameter of $\Omega$.  Then $AD'=156$; combining this with $AB=65$ yields $BD'=169$ as well.  From here it's not hard to see that $D\equiv D'$, so $\triangle DAB$ is a right triangle.

\par Finally, note that since $A$ and $D$ are antipodal $\angle DCA=90^\circ$ as well. Thus $C$ is the foot of the perpendicular from $A$ to $BD$, so \[AC=\dfrac{AB\cdot AD}{BC}=\dfrac{65\cdot 156}{169}=\boxed{60}.\]

\item Identical spherical tennis balls of radius 1 are placed inside a cylindrical container of radius 2 and height 19. Compute the maximum number of tennis balls that can fit entirely inside this container.

\proposed{Patrick Lin}

\solution Observe that we can fit two balls into the bottom such that they both touch the bottom; it is then clear that the optimal way to pack in balls is to place them in layers of two each, then stack them in such a way that the line formed by connecting the centers of each pair is orthogonal to the pair above and below.

\par Since each ball is tangent to the other three, connecting their centers forms a tetrahedron $ABCD$ of side length 2. Let $M$ be the midpoint of $AB$ and $N$ be the midpoint of $CD$. To find the vertical height between the two layers, it suffices to compute $MN$. Project $D$ and $N$ onto the plane $ABC$ to points $D'$ and $N'$, respectively, then we find that $CN' = N'D' = DM$, and thus $MN' = \tfrac{2\sqrt{3}}{3}$. Further, by Pythagorean Theorem we have $DD' = \tfrac{2\sqrt{6}}{3}$ and $NN' = \tfrac12 DD' = \tfrac{\sqrt{6}}{3}$. Hence $MN = \sqrt{MN'^2 + NN'^2} = \sqrt{2}$.

\par Let $k$ be the maximum number of layers we can fit inside this container; this is the maximum solution to the inequality $2 + (k-1)\sqrt{2} \le 19$. Noting that $\tfrac{17}{\sqrt{2}} = \sqrt{144.5} > 12$, we have $k = 13$, and so the answer is $\boxed{26}$.

\item Triangle $ABC$ satisfies $AB=28$, $BC=32$, and $CA=36$, and $M$ and $N$ are the midpoints of $\overline{AB}$ and $\overline{AC}$ respectively.  Let point $P$ be the unique point in the plane $ABC$ such that $\triangle PBM\sim\triangle PNC$.  What is $AP$? %David Altizio

\solution Scale down by a factor of $4$, so that $AB=7$, $BC=8$, and $CA=9$.

\par Note that $P$ is the intersection of the circumcircles of $\triangle ANB$ and $\triangle AMC$.  To see this, remark that by the similarity condition $\angle BMP=\angle NCP$, so quadrilateral $AMQC$ is cyclic.  Similarly, $ANPB$ is also cyclic.

\par Now perform a $\sqrt{bc}$ inversion $\Phi$ about $A$.  Note that $\Phi$ sends $B$ to $C$ and vice versa.  Furthermore, $\Phi$ sends $M$ to the point $M'$ on ray $\overrightarrow{AC}$ such that \[AM\cdot AM'=AB\cdot AC\quad\implies\quad AM'=2AC.\] Similarly, this inversion sends $N$ to the point $N'$ such that $AN'=2AB$.  As a result, $\Phi$ sends $(ANB)$ to $CN'$ and $(AMC)$ to $BM'$, meaning that the image of $Q$ about $\Phi$ is the centroid of the triangle homothetic to $\triangle ABC$ with scale factor $2$ (i.e. $\triangle AN'M'$).

\par The rest is computation.  Let $m_a$ be the length of the $A$-median of $\triangle ABC$.  Then \[AP'=\frac43m_a=\dfrac43\sqrt{\dfrac{2(AB^2+AC^2)-BC^2}4}=\frac{28}3.\] Therefore, by the definition of inversion, \[AP'\cdot AP = AB\cdot AC = 63\quad\implies\quad AP=\frac{63}{28/3}=\frac{27}4.\] Scaling back up by a factor of $4$ yields the desired answer of $\boxed{27}$.

\proposed{David Altizio}

\end{enumerate}

\newpage

\begin{center}
\huge\textbf{Number Theory Tiebreaker}\normalsize

\vspace{3pt}
\end{center}

\begin{enumerate}

\item For all integers $n\geq 2$, let $f(n)$ denote the largest positive integer $m$ such that $\sqrt[m]{n}$ is an integer.  Evaluate \[f(2)+f(3)+\cdots+f(100).\] %Cody Johnson

\proposed{Cody Johnson}

\solution Increment once for each perfect power. There are $100-1=99$ first powers, $\sqrt{100}-1=9$ squares, $\lfloor\sqrt[3]{100}\rfloor-1=3$ cubes, etc. for a total of $99+9+3+2+1+1=\boxed{115}$.

\item For each integer $n\geq 1$, let $S_n$ be the set of integers $k > n$ such that $k$ divides $30n-1$.  How many elements of the set \[\mathcal{S} = \bigcup_{i\geq 1}S_i = S_1\cup S_2\cup S_3\cup\ldots\] are less than $2016$?

\proposed{Cody Johnson}

\solution Note that if $k\mid30n-1$ then $\gcd(30,k)=1$. Now if $\gcd(30,k)=1$ then let $n=30^{-1}\pmod{k}$. We have $k\mid30n-1$ and $k>n$, so $\mathcal S=\{a\mid\gcd(a,30)=1\}$. Finally, since $\phi(30)=8$, we can look at just the first $2016\pmod{30}$ numbers and add them in. Since $1,5$ are the only numbers in $[1,6]$ coprime with $30$, the answer is $8\cdot\frac{2016-6}{30}+2=\boxed{538}$.

\item Let $\{x\}$ denote the fractional part of $x$.  For example, $\{5.5\}=0.5$.  Find the smallest prime $p$ such that the inequality \[\sum_{n=1}^{p^2}\left\{\dfrac{n^p}{p^2}\right\}>2016\] holds.

\proposed{Andrew Kwon}

\solution For each $ 1 \leq n \leq p^{2}$, let $n = kp + \ell$, with $0 \leq k,
\ell \leq p-1$. Note that this fractional part is equivalent to the sum of the
remainders when $n^{p}$ is divided by $p^{2}$. Then,
\[
	n^{p} \equiv (kp + \ell)^{p} \equiv \dbinom{p}{1} (kp)(\ell)^{p-1} +
	\dbinom{p}{0} \ell^{p} \pmod{p^{2}}.
\]
That is, $n^{p} \equiv \ell^{p} \pmod{p^{2}}$. Then, for each $n \geq 1$ not
divisible by $p$,
\[
	\left\{ \frac{n^{p}}{p^{2}} \right\} + \left\{
		\frac{(p^{2}-n)^{p}}{p^{2}}\right\} = 1,
\]
while there are $p^{2} - p$ such values of $n$ from $1$ to $p^{2}$. Thus, 
\[
	\sum_{n=1}^{p^{2}} \left\{\frac{n^{p}}{p^{2}}\right\} =
	\frac{p^{2}-p}{2},
\]
and it follows that $p^{2} - p > 4032$. Simple estimates yield $p =\boxed{67}$ as the
smallest possible prime.


\end{enumerate}

\end{document}

