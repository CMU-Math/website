\documentclass[10pt]{article}
\usepackage{amsmath, amssymb, amsthm}
\newcommand{\abs}[1]{\lvert #1 \rvert}
\newtheorem{lemma}{Lemma}
\usepackage[top=2cm, left = 2cm, right = 2cm, bottom = 3cm]{geometry}
\usepackage[pdftex]{graphicx}
\usepackage{asymptote}
\usepackage{fancyhdr}
\newcommand{\N}{\mathbb{N}}
\pagestyle{fancy}
\rhead{}
\chead{\includegraphics[scale=0.1]{../CMIMC-header-2018.png}}
\lhead{}
\setlength{\headheight}{43pt}
\rfoot{}
\cfoot{}
\lfoot{}
\newcommand{\proposed}[1]
{
\vspace{5pt}
\noindent\textit{Proposed by #1}
}
\newcommand{\solution}
{
\vspace{5pt}
\noindent\textit{Solution.}\qquad
}
\begin{document}

\begin{center}
\huge\textbf{Team Solutions Packet}\normalsize

\vspace{3pt}
\end{center}

\begin{enumerate}

% G - A
% T = 61; ANS = 44
\item[1-1.] Let $ABC$ be a triangle with $BC=30$, $AC=50$, and $AB=60$.  Circle $\omega_B$ is the circle passing through $A$ and $B$ tangent to $BC$ at $B$; $\omega_C$ is defined similarly.  Suppose the tangent to $\odot(ABC)$ at $A$ intersects $\omega_B$ and $\omega_C$ for the second time at $X$ and $Y$ respectively.  Compute $XY$.

\proposed{David Altizio}

\solution
For simplicity, let $BC=a$, $AC=b$, and $AB=c$.  Note that angle chasing yields \[\angle XAB = \angle ACB = \angle AYC\qquad\text{and}\qquad \angle YAC = \angle ABC = \angle AXB,\] so $\triangle BXA\sim\triangle ABC\sim\triangle CAY$.  This in particular implies \[\dfrac{AY}{a} = \dfrac{b}{c}\qquad\text{and}\qquad \dfrac{AX}{a} = \dfrac{c}{b},\] and so \[XY=AX+AY = a\left(\frac bc+\frac cb\right) = 30\left(\frac 56+\frac65\right) = \boxed{61}.\]

\begin{center}
\includegraphics[scale=0.8]{proposal_asy.PNG}
\end{center}

\item[1-2.] Let $T = TNYWR$. For some positive integer $k$, a circle is drawn tangent to the coordinate axes such that the lines $x + y = k^2, x + y = (k+1)^2, \dots, x+y = (k+T)^2$ all pass through it. What is the minimum possible value of $k$?

\proposed{Patrick Lin}

\solution
It suffices to consider the set of circles that contain a tangent line $x + y = k^2$ on the bottom-left of the circle. For a fixed $k$, consider the corresponding circle of radius $r$; we see that $\tfrac{k^2}{2} = (\sqrt{2} - 1)r$. The maximum $K$ such that the line $x + y = K^2$ passes through the circle is the maximal solution to $\tfrac{K^2}{2} \le (\sqrt{2} + 1)r \iff K \le (\sqrt{2} + 1)k$. Hence a total of $\lfloor k\sqrt{2} \rfloor + 1$ lines pass through this circle. Returning to the original problem, it's clear that we then need $k \ge \frac{T}{\sqrt{2}}$ in order to pass through $T + 1$ lines; substituting $T = 61$ (and approximating $\sqrt{2} \approx 1.4$) yields an answer of $\boxed{44}$.


% A - A
% T = 117; ANS = 2^25
\item[2-1.] Suppose that $a$ and $b$ are non-negative integers satisfying $a + b + ab + a^b = 42$. Find the sum of all possible values of $a + b$.

\solution
We first case on $a$: if $a = 0$ then we immediately get $b = 42$, and if $a = 1$ then $b = 20$. If $a = 2$, then we have $3b + 2^b = 42$, which can't be satisfied since $2^b$ cannot be a multiple of 3. Hence, for all remaining solutions, $a \ge 3$, and so $b \le 3$, else $a^b > 42$. Trying the remaining values of $b$ give $(a,b) = (3,3),(5,2),(41,0)$, for an answer of $42+21+6+7+41=\boxed{117}$.


\item[2-2.] Let $T = TNYWR$. Suppose that a sequence $\{a_n\}$ is defined via $a_1 = 11, a_2 = T$, and $a_n = a_{n-1} + 2a_{n-2}$ for $n \ge 3$. Find $a_{19} + a_{20}$.

\proposed{Keerthana Gurushankar}

\solution
Note that \[a_n + a_{n-1} = a_{n-1} + 2a_{n-2} + a_{n-1} = 2(a_{n-1}+a_{n-2}) = \dots = 2^{n-2}(a_2 + a_1).\] Substituting $n = 20$ gives $a_{20} + a_{19} = 2^{18}(11 + T)$, and substituting $T = 117$ yields an answer of $\boxed{2^{25}}$.


% G - C
% T = 6; ANS = 1/455
\item[3-1.] Let $X$ and $Y$ be points on semicircle $AB$ with diameter 3. Suppose the distance from $X$ to $AB$ is $\frac{5}{4}$ and the distance from $Y$ to $AB$ is $\frac{1}{4}$. Compute \[(AX+BX)^2 - (AY+BY)^2.\]

\proposed{David Altizio}

\solution Note $AX^2 + BX^2 = AY^2 + BY^2 = 9$ and further that $AX \cdot BX = 3\cdot\frac54$ by different area calculations. The desired quantity then reduces to \[2\left(AX \cdot BX - AY \cdot BY\right) = 2\left(3\cdot\frac54 - 3\cdot\frac14\right) = \boxed{6}.\]


\item[3-2.] Let $T = TNYWR$. $T$ people each put a distinct marble into a bag; its contents are mixed randomly and one marble is distributed back to each person. Given that at least one person got their own marble back, what is the probability that everyone else also received their own marble?

\proposed{Patrick Lin}

\solution
Let $A$ be the event that everybody gets their marble back, and $B$ be the event that at least one person gets their marble back. Then Bayes suggests that
\[\Pr[A \mid B] = \frac{\Pr[B \mid A]\cdot\Pr[A]}{\Pr[B]}.\]
Clearly, $\Pr[B \mid A] = 1$ and $\Pr[A] = \tfrac{1}{T!}$. Finally, $\Pr[B] = 1 - \frac{!n}{n!}$, where $!n$ equals the number of derangements on the set $[n]$. Using either the recurrence $!n = n \cdot !(n-1) + (-1)^n$ or the recurrence $!n = (n-1)(!(n-1) + !(n-2))$ or the identity $!n = \lfloor \frac{n!}{e}\rfloor$, we find $!T = 265$, and hence the answer is $\frac{1}{T!-!T} = \boxed{\tfrac{1}{455}}$.


% C - G
% T = 32 [232]; ANS = 232/7
\item[4-1.] Define an integer $n \ge 0$ to be \textit{two-far} if there exist integers $a$ and $b$ such that $a$, $b$, and $n + a + b$ are all powers of two. If $N$ is the number of two-far integers less than 2048, find the remainder when $N$ is divided by 100.

\proposed{Patrick Lin}

\solution Write $a = 2^x$, $b = 2^y$, and $n + a + b = 2^z$ for some $x,y,z\in\mathbb N$.  Then the condition rearranges to \[n = 2^z - 2^x - 2^y.\] Note that $n=0$ trivially works, so assume $n>0$.  We now claim that if $n$ can be written in this form, then it can be written in such a form where $x$, $y$, and $z$ are pairwise distinct.  In particular, if not all of $x$, $y$, and $z$ are pairwise distinct, then $x = y < z-1$ (else $n$ is nonpositive), and so \[2^z - 2^x - 2^y = 2^z - 2^{x+1} = 2^{z+1} - 2^z - 2^{x+1}.\] Now we claim that there exists a unique representation of $n$ in this form when $x,y,z$ are all pairwise distinct up to permutation of $x$ and $y$.  To prove this, write \[n = 2^{z_0} - 2^{x_0} - 2^{y_0} = 2^{z_1} - 2^{x_1} - 2^{y_1}\quad\Rightarrow\quad 2^{z_0} + 2^{x_1} + 2^{y_1} = 2^{z_1} + 2^{x_0} + 2^{y_0}.\] Since binary representations of numbers are unique (due to $x_i,y_i,z_i$ all being distinct), it must be the case that $\{x_0,y_0,z_0\} = \{x_1,y_1,z_1\}$.  But $z_0$ and $z_1$ are the largest numbers in their respective sets, so $z_0 = z_1$ and $\{x_0,y_0\} = \{x_1,y_1\}$ as desired.

\par It now suffices to count the number of triples $(x,y,z)$ which give a positive $n < 2048$, where WLOG assume $x>y$ for simplicity.  Note that if $z \leq 11$, then in fact any triple of integers works, and so the answer in this case is just $\binom{12}3 = 220$.  If $z = 12$, then it must be the case that $y = 11$, or else $n$ is too large; there are thus $11$ cases here, corresponding to $x\in\{0,\ldots, 10\}$.  Finally, if $z > 12$, then it is easy to see that $n$ is always too large, so no cases exist.  Adding back the $1$ to deal with $n=0$ gives a final answer of $220 + 11 + 1 = 232$; the last two digits of this are $\boxed{32}$.


\item[4-2.] Let $T = TNYWR$. Let $CMU$ be a triangle with $CM=13$, $MU=14$, and $UC=15$.  Rectangle $WEAN$ is inscribed in $\triangle CMU$ with points $W$ and $E$ on $\overline{MU}$, point $A$ on $\overline{CU}$, and point $N$ on $\overline{CM}$.  If the area of $WEAN$ is $T$, what is the largest possible value for its perimeter?

\proposed{David Altizio}

\solution Let $WE=AN=\ell$.  Now pick $U'\in MU$ such that $NU'\parallel CU$.  Then $\triangle NMU'\sim\triangle CMU$, and in particular \[MU' = MW + WU' = MW + EU = 14-\ell.\] Thus $NW = \frac{12}{14}(14-\ell) = 12 - \frac67\ell$, and so \[T = NW\cdot WE = \ell\left(12 - \frac67\ell\right).\] Plugging in $T=32$ and solving the quadratic for $\ell$ yields $\ell = 7\pm\sqrt{\frac{35}3}$, and so the maximum possible value for the perimeter of $WEAN$ is \[2\left[\ell + \left(12 - \frac67\ell\right)\right] = 24 + \frac27\ell = \boxed{26 + \frac{\sqrt{440}}{21}}.\]


% N - C
% T = 32; ANS = 1/2^31
\item[5-1.] How many ordered triples $(a,b,c)$ of integers satisfy the inequality \[a^2+b^2+c^2 \leq a+b+c+2?\]

\proposed{David Altizio}

\solution The condition is equivalent to \[(2a-1)^2 + (2b-1)^2 + (2c-1)^2 \leq 11.\] Now note that if $|2a-1| = 3$, then the only way we can satisfy the inequality is if $|2b-1| = |2c-1| = 1$.  Thus it must be the case that at least two of $a,b,c$ are equal to either $0$ or $1$, and the third can either be equal to $0$, $1$, $2$, or $-1$.  A quick count gives $2^3 + 3\cdot 2^3 = \boxed{32}$ solutions.

\item[5-2.] Let $T = TNYWR$. David rolls a standard $T$-sided die repeatedly until he first rolls $T$, writing his rolls in order on a chalkboard. What is the probability that he is able to erase some of the numbers he's written such that all that's left on the board are the numbers $1, 2, \dots, T$ in order?

\proposed{Patrick Lin}

\solution
Let $p_k$ be the probability that, given a die which rolls outcomes from the set $\{k, k+1, \dots, T\}$, David is able to obtain the sequence $k, k+1, \dots, T$. Clearly $p_T = 1$ and $p_{T-1} = \frac12$. Note that in general we have $p_k = \frac12 p_{k+1}$, since we must roll $k$ before rolling $T$, and after that the event corresponds precisely to that of $p_{k+1}$. Hence, $p_1 = \frac{1}{2^T} = \boxed{\tfrac{1}{2^{31}}}$.


% C - C
% T: 151 [151/360]; ANS: 10950
\item[6-1.] Jan rolls a fair six-sided die and calls the result $r$. Then, he picks real numbers $a$ and $b$ between 0 and 1 uniformly at random and independently. If the probability that the polynomial $f(x) = \tfrac{x^2}{r} - x\sqrt{a} + b$ has a real root can be expressed as simplified fraction $\frac{p}{q}$, find $p$.

\proposed{Patrick Lin}

\solution
Observe that $f$ has a real root if and only if the discriminant is non-negative, which rearranges to the condition $r\cdot a \ge 4b$. Casing on the value of $r$, we obtain that the probability is \[\frac{1}{6}\left(\frac{1}{8}+\frac{2}{8}+\frac{3}{8}+\frac{4}{8}+\frac{6}{10}+\frac{8}{12}\right) = \frac{151}{360},\] and so the desired answer is $\boxed{151}$.


% a little easier but should be fine
\item[6-2.] Let $T = TNYWR$. Compute the number of ordered triples $(a,b,c)$ such that $a$, $b$, and $c$ are distinct positive integers and $a + b + c = T$.

\proposed{Patrick Lin}

\solution Assume that $a < b < c$; then we may reparametrize $b = a + x$ and $c = a + x + y$ for $x, y > 0$, and the desired condition becomes $3a + 2x + y = T$. For every choice of $a$ and $x$ such that $3a + 2x < T$ we have exactly one solution, and so this gives an answer of
\[\sum_{a=1}^{\left\lfloor\frac{T}{3}\right\rfloor - 1} \sum_{x=1}^{\left\lfloor\frac{T - 3a}{2}\right\rfloor - 1} 1 = \sum_{a=1}^{\left\lfloor\frac{T}{3}\right\rfloor - 1} \left(\left\lfloor\frac{T - 3a}{2}\right\rfloor - 1\right).\]
Substituting $T = 151$ we find that the sum is equal to $73 + 72 + 70 + 69 + \dots + 3 + 1 = 1825$. Finally, multiplying by six to account for our original assumption yields $\boxed{10950}$.

\begin{center}OR\end{center}

\solution Note that by Stars and Bars the number of solutions without the distinct condition is $\binom{150}2 = 11175$.  To compute the number of solutions with some of $a,b,c$ equal to each other, note that $a=b=c$ is not possible since $3\nmid 151$, so it suffices to compute the number of solutions to $2a+b=151$ and multiply by $3$.  Here, all $b$ odd between $1$ and $149$ generate a solution, for $75$ possible pairs.  Thus the requested answer is $11175 - 3\cdot 75 = \boxed{10950}$.


% G - C
% T: 15 [4/11]; ANS: 98?
\item[7-1.] Let $ABCD$ be a unit square, and suppose that $E$ and $F$ are on $\overline{AD}$ and $\overline{AB}$ such that $AE = AF = \tfrac23$. Let $\overline{CE}$ and $\overline{DF}$ intersect at $G$. If the area of $\triangle CFG$ can be expressed as simplified fraction $\frac{p}{q}$, find $p + q$.

\proposed{Patrick Lin}

\solution
The most straightforward solution is simply to use coordinates. We have $C = (1,0)$, $F = (\tfrac23, 1)$, and can compute $G$ to be $(\tfrac{2}{11},\tfrac{3}{11})$ by intersecting lines $x + 3y = 1$ and $3x = 2y$. Using the shoelace theorem yields an area of $\tfrac{4}{11}$, so the answer is $\boxed{15}$.


\item[7-2.] Let $T = TNYWR$. A total of $2T$ students go on a road trip. They take two cars, each of which seats $T$ people. Call two students \textit{friendly} if they sat together in the same car going to the trip and in the same car going back home. What is the smallest possible number of friendly pairs of students on the trip?

\proposed{Cody Johnson}

\solution Number the cars $C_1$ and $C_2$.  Denote by $n$ the number of students who sat in car $C_1$ during both trips.  Then $T - n$ students sat in car $C_1$ the first trip and car $C_2$ the second trip.  This means that there must have been $T - n$ students starting in car $C_2$ but then moving on to $C_1$, which finally implies there are $n$ students who stayed put in car $C_2$ during both trips.  The number of friendly pairs is thus \[2\binom n2 + 2\binom{T-n}2 = n(n-1) + (T-n)(T-n-1) = 2n^2 - 2nT + T^2 - T;\] plugging in $T = 15$ implies this simplifies to $2n^2 - 30n + 210$.  Now this is a quadratic in $n$ and so it is maximized by taking $n$ near the vertex of the parabola, which occurs at $n=\tfrac{15}2$.  Thus the minimum is taken from either $n=7$ or $n=8$ and has value $\boxed{98}$.

% G - A
% T = (3, 2) [32]; ANS = 9602
\item[8-1.] Let $\triangle ABC$ be a triangle with $AB=3$ and $AC=5$. Select points $D, E,$ and $F$ on $\overline{BC}$ in that order such that $\overline{AD}\perp \overline{BC}$, $\angle BAE=\angle CAE$, and $\overline{BF}=\overline{CF}$. If $E$ is the midpoint of segment $\overline{DF}$, what is $BC^2$?

\proposed{Fei Peng}

\solution
First, by the Angle Bisector Theorem, set $BE=3k$ and $CE=5k$ for some constant $k$ so that $BF=CF=4k$; by the condition, we derive that $BD=2k$ and $CD=6k$. Now let $BC=a$ such that by the Law of Cosines, $$\frac{BD}{CD}=\frac{3\cos B}{5\cos C}=\frac{3}{5}\cdot \frac{\frac{a^2+3^2-5^2}{2\cdot a\cdot 3}}{\frac{a^2-3^2+5^2}{2\cdot a\cdot 5}}=\frac{a^2-16}{a^2+16}=\frac{1}{3}$$ from which we derive $a^2=\boxed{32}$. 


\item[8-2.] Let $T = TNYWR$, and let $T = 10X + Y$ for an integer $X$ and a digit $Y$. Suppose that $a$ and $b$ are real numbers satisfying $a+\frac1b=Y$ and $\frac{b}a=X$. Compute $(ab)^4+\frac1{(ab)^4}$.

\proposed{Cody Johnson}

\solution From the previous solution, we see that $X = 3$ and $Y = 2$.  Note that $6 = XY = \tfrac ba\cdot (a+\tfrac1b) = b + \tfrac1a$.  This means that \[12 = \left(a + \frac1b\right)\left(b+\frac1a\right) = ab + 2 + \frac 1{ab}\quad\Rightarrow\quad ab + \frac 1{ab} = 10.\] Now squaring this yields \[100 = \left(ab + \frac{1}{ab}\right)^2 = (ab)^2 + 2 + \frac{1}{(ab)^2}\quad\Rightarrow\quad (ab)^2 + \frac{1}{(ab)^2} = 98,\] and performing this operation one last time yields $(ab)^4 + \tfrac{1}{(ab)^4} = 98^2 - 2 = \boxed{9602}$.

% C - A
% T: 337 [256/81]; ANS: \sqrt{1347}
\item[9-1.] Andy rolls a fair 4-sided dice, numbered 1 to 4, until he rolls a number that is less than his last roll. If the expected number of times that Andy will roll the dice can be expressed as a reduced fraction $\frac{p}{q}$, find $p + q$.

\proposed{Eric Chen}

\solution
Let $E_k$ be the expected number of additional times Andy will roll the dice given that his last roll was $k$. Then we have the recursive relations $E_4 = 1 + \frac14 E_4$, $E_3 = 1 + \frac14 (E_3 + E_4)$, $E_2 = 1 + \frac14 (E_2 + E_3 + E_4)$, and $E_1 = 1 + \frac14 (E_1 + E_2 + E_3 + E_4)$. We may solve this system to obtain $E_4 = \frac{4}{3}$, $E_3 = \frac{16}{9}$, $E_2 = \frac{64}{27}$, and $E_1 = \frac{256}{81}$. His initial roll is as if he had last rolled 1, and so the desired answer is $256+81=\boxed{337}$.


\item[9-2.] Let $T = TNYWR$. The solutions in $z$ to the equation \[\left(z + \frac Tz\right)^2 = 1\] form the vertices of a quadrilateral in the complex plane.  Compute the area of this quadrilateral.

\proposed{David Altizio}

\solution By multiplying through by $z^2$, the equation rewrites as \[(z^2 + T)^2 = z^2\quad\Rightarrow\quad (z^2 - z + T)(z^2 + z + T) = 0.\] Solving yields $z = \tfrac{\pm 1\pm \sqrt{1-4T}}2$.  Since $T$ is an integer, the roots must be imaginary, and in particular the quadrilateral they form is a rectangle.  Its area is thus \[\left(\frac 12 - \left(-\frac12\right)\right)\left(\frac{\sqrt{4T-1}}2 - \left(-\frac{\sqrt{4T-1}}2\right)\right) = \sqrt{4T-1} = \boxed{\sqrt{1347}}.\]

% A - G
% T: 36; ANS: 128/21
\item[10-1.] Find the smallest positive integer $k$ such that $\underbrace{11\dots11}_{\text{$k$ 1's}}$ is divisible by 9999.

\proposed{Patrick Lin}

\solution
We wish to find the smallest $k$ such that $\frac{10^k-1}{9} \equiv 0 \mod{9999}$. This may be rewritten into $\underbrace{11\dots11}_{\text{$k$ 1's}} \equiv 0 \mod 9$, $10^k \equiv 1 \mod 11$, and $10^k \equiv 1 \mod 101$. The first condition yields $9 \mid k$, the second gives $2 \mid k$, and the third gives $4 \mid k$. Taking the least common multiple yields $\boxed{36}$.

\begin{center}OR\end{center}

\solution
We wish to find the smallest $k$ such that \[\frac{11\cdots 11}{9999} = \frac{10^k - 1}{9(10^4 - 1)}\in\mathbb Z.\] Note that since $10^4 - 1\mid 10^k - 1$, we must have $4\mid k$.  Let $k=4k_0$ for some $k_0\in\mathbb Z$.  Then \[\frac{10^{4k_0} - 1}{9(10^4 - 1)} = \frac{1 + 10^4 + \cdots + 10^{4(k_0 - 1)}}9.\] Now remark that the numerator is congruent to $1+1+\cdots + 1 \equiv k_0 \mod 9$, so we need $9\mid k_0$.  Thus the minimum $k$ is $9\cdot 4 = \boxed{36}$.


\item[10-2.] Let $T = TNYWR$. Circles $\omega_1$ and $\omega_2$ intersect at $P$ and $Q$.  The common external tangent $\ell$ to the two circles closer to $Q$ touches $\omega_1$ and $\omega_2$ at $A$ and $B$ respectively.  Line $AQ$ intersects $\omega_2$ at $X$ while $BQ$ intersects $\omega_1$ again at $Y$.  Let $M$ and $N$ denote the midpoints of $\overline{AY}$ and $\overline{BX}$, also respectively.  If $AQ=\sqrt{T}$, $BQ=7$, and $AB=8$, then find the length of $MN$.

\proposed{David Altizio}

\solution Note that since $PXAQ$ and $PYBA$ are cyclic quadrilaterals, $\angle PXQ=\angle PAQ$ and $\angle PBQ=\angle PYQ$, so $\triangle PXB\sim\triangle PAY$.  By considering the spiral similarity sending the former triangle to the latter, we deduce that $\triangle PXA\sim\triangle PBY$.  (This can also be shown via simple angle chasing.)  Note that $M$ and $N$ are corresponding points on these triangles, so $\triangle PXM\sim\triangle PBN$, which means that \[\frac{PX}{PM}=\frac{PB}{PN}\quad\implies \quad \frac{PX}{PM}=\frac{PB}{PN}.\]  Combining this with the fact that $\angle XPB=\angle MPN$ yields that $\triangle PXB\sim\triangle PMN$.

\par Now note that $\triangle PXM\sim\triangle AQT$ as well, where $T=PQ\cap AB$.  To see this, construct $C$ such that $AQBC$ is a parallelogram (so that $T$ is the intersection point of the two diagonals).  Then $\angle AQC=\angle PXA$ and \[\angle ACQ=\angle CQB=\angle PQX=\angle PQX,\] so $\triangle PXA\sim\triangle AQT$.  From the fact that $T$ and $M$ are both corresponding points in these two triangles, we obtain the desired conclusion.

\par As a result, simple computation gives \[MN = \dfrac{PM}{PX}\cdot XB = \dfrac{AT}{AQ}\cdot XB = \dfrac{AB^3}{2\cdot AQ\cdot QB} = \dfrac{8^3}{2\cdot 6\cdot 7} = \boxed{\dfrac{128}{21}}.\]

\begin{center}OR\end{center}

\solution
Let $R$ be the midpoint of $\overline{AB}$.  Then $MR=\frac12BY$, $RN=\frac12AX$, and $\angle MRN=\angle AQB$ since lines $MR$, $RN$, $BQ$, and $AQ$ form a parallelogram.  We have \[BY=\frac{AB^2}{BQ}=\frac{64}7\qquad\text{and}\qquad AX=\frac{AB^2}{AQ}=\frac{32}3,\] so $MR=\frac{32}7$ and $RN=\frac{16}3$.  Also, \[\cos\angle MRN = \cos\angle AQB = \dfrac{-8^2+6^2+7^2}{2\cdot 6\cdot 7} = \dfrac14.\] Thus \begin{align*}MN^2 &= \left(\dfrac{32}7\right)^2 + \left(\dfrac{16}3\right)^2 - 2\left(\dfrac{32}7\right)\left(\dfrac{16}3\right)\left(\dfrac14\right)\\&=\left(\dfrac{16}{21}\right)^2\left(6^2+7^2 - 21\right) = \left(\dfrac{16}{21}\right)^2\cdot 64,\end{align*} so $MN=\tfrac{16}{21}\cdot 8 = \boxed{\tfrac{128}{21}}$.

    
\end{enumerate}

\end{document}
